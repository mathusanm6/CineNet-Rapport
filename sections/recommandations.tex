\section{Recommandations}

Les méthodes utilisées pour la recommandation de films, de posts et d'évènements sont identiques sur le plan technique. En effet, les recommandations sont basées sur les goûts des utilisateurs, déterminés par les films qu'ils ont notés, les posts qu'ils ont aimés et les évènements auxquels ils ont participé.

\subsection{Exemple : Recommandation de films}
Tout d'abord, nous calculons la similarité entre les utilisateurs. Par exemple, les similarités entre les films sont calculées en utilisant une formule basée sur le produit scalaire des notes des utilisateurs. Cela implique de comparer les notes données par les mêmes utilisateurs pour différents films.

Dans le cas de la recommandation de films, pour assurer que les similarités calculées sont significatives, seules les paires de films notées par plus d'un utilisateur sont retenues. Cela élimine les cas où la similarité serait basée sur des données insuffisantes.

Les scores de recommandation pour chaque utilisateur et chaque film non encore noté sont ensuite calculés en combinant les similarités des films notés avec les notes données par l'utilisateur. Cela se fait via une somme pondérée des similarités et des notes. Les films déjà notés par l'utilisateur sont exclus de ce calcul pour ne recommander que des films nouveaux.

Les recommandations calculées sont ensuite insérées dans une table permanente \textit{MovieRecommendation}. En cas de conflit (si une recommandation existe déjà pour un utilisateur et un film), la recommandation est mise à jour avec le nouveau score calculé.
