\subsection{Requête SQL pour le Calcul de la Moyenne du Nombre Maximum de Réactions par Publication}

Cette requête SQL calcule la moyenne du nombre maximum de réactions par publication. La moyenne est calculée en groupant d'abord les réactions par post\_id et en comptant le nombre de réactions par publication. 
Le nombre maximum de réactions par publication est ensuite calculé en groupant les comptes par post\_id et en prenant le compte maximum. 
Enfin, la moyenne des réactions maximum par publication est calculée en prenant la moyenne des comptes maximums.

\begin{lstlisting}
SELECT
    AVG(Max_Reactions) AS Average_of_Max_Reactions
FROM (
    SELECT
        post_id,
        MAX(reactions_count) AS Max_Reactions
    FROM (
        SELECT
            post_id,
            COUNT(*) AS reactions_count
        FROM
            Reactions
        GROUP BY
            post_id) AS ReactionCounts
    GROUP BY
        post_id) AS MaxCounts;
\end{lstlisting}